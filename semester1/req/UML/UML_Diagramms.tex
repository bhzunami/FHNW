\documentclass[12pt,article]{memoir}
\usepackage[utf8]{inputenc}
\usepackage[T1]{fontenc}
\usepackage{mathpazo}
\linespread{1.05}
\usepackage{tikz}
\usepackage{tikz-uml} 
\usepackage{amsmath}
\usepackage{savetrees}
\begin{document}
\begin{center}
\begin{figure}[h!]
\begin{tikzpicture}
\begin{umlsystem}{Autoverwaltung Nutz AG}
% usecase-1
\umlusecase[x=-1]{Anzeigen von verfügbaren Fahrzeuge}
%
% usecase-2
\umlusecase[x=1,y=2]{Drucken von Einzahlungsscheine}
%
% usecase-3
\umlusecase[x=-1,y=4]{Anzeigen von reservierten Fahrzeugen}
%
% usecase-4
\umlusecase[x=1, y=6]{Reparatur fällige Fahrzeuge anzeigen}
%
% usecase-5
\umlusecase[x=-1,y=8]{Statistik von jedem Fahrzeuge}
%

\end{umlsystem}

\umlactor[x=-8,y=2]{Disposition}
\umlactor[x=-8,y=4]{Management}
\umlactor[x=8,y=4]{System}

\umlassoc{Disposition}{usecase-1}
\umlassoc{Disposition}{usecase-3}

\end{tikzpicture}
\caption{Example of Use case diagram using tikz-uml. Examlpe from the book Requirements Engineering by Axel van Lamsweerde.}
\end{figure}
\end{center}
\end{document}