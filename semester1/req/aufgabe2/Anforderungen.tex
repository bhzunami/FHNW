%
% Authors
% Denis Augsburger
% Piero Steinger
% Thomas Wilde
% Nicolas Mauchle
% 
%
% Version 1.0
%
\documentclass[12pt,a4paper,german]{article}
%
\author{Denis Augsburger, Piero Steinger, Thomas Wilde, Nicolas Mauchle }
%
\usepackage[left=2.5cm,right=2.5cm,bottom=2.5cm,includeheadfoot]{geometry}
\usepackage{babel}
\usepackage[utf8]{inputenc}
\usepackage{fancyhdr}
\usepackage{lastpage}
\usepackage{hyperref}
\hypersetup{
    pdfborder = {0 0 0}
}
\usepackage[all]{hypcap}
\usepackage[stable]{footmisc}
% KOPF UND FUSS ZEILEN
\pagestyle{fancy}
%
\fancyhf[R]{}
\fancyhf[L]{\leftmark}
%
% suppress page number in bottom center:
\cfoot{}
%
\fancyfoot[L]{}
\fancyfoot[R]{Seite \thepage \  von  \pageref{LastPage}}
%
%Linie unten
\renewcommand{\footrulewidth}{0.5pt}
%
\newcommand{\HRule}{\rule{\linewidth}{0.5mm}}
%
% Document
%
\begin{document}
\begin{titlepage}
\begin{center}
\textsc{\LARGE Fachhochschule Nordwestschweiz}\\[1.5cm]
% Title
\HRule \\[0.4cm]
{ \huge \bfseries Portfolio}\\[0.4cm]
\HRule \\[1.5cm]
% Author and supervisor
\begin{minipage}{1.2\textwidth}
\begin{flushleft} \large
\emph{Autor:}\\
Lea Knöll
\newline
\newline
\emph{Dozent:} \\
René Fankhauser
\newline
\newline
\emph{Klasse:} \\
3. Semester\\
%\end{flushright}
\end{flushleft}
\end{minipage}
\vfill
% Bottom of the page
{\large \today}
\end{center}
\end{titlepage}

\newpage
\section{Ziele}
Die folgenden Ziele sind mit Hilfe der Satzschablone definiert worden.
\begin{itemize}
\item Das System muss dem Kunden die Möglichkeit bieten, ein Fahrzeug zu reservieren/mieten.
%
\item Wurde ein Fahrzeug von einem Kunden reserviert, muss das System der Disposition die Möglichkeit bieten, Informationen zur Reservationen anzuzeigen, damit das Fahrzeug bereitgestellt werden kann.
%
\item Das System soll fähig sein eine Partner-Werkstatt für einen Reparaturauftrag zu kontaktieren.
%
\item Das System muss fähig sein, aus der Buchhaltung heraus eine Rechnung für einen Kunden zu erstellen.
%
\item Wir eine Rechnung nicht in der abgemachten Frist bezahlt, soll das System die Buchhaltung darauf Hinweisen. 
%
\item Das System muss fähig sein, eine Werkstatt für eine Reparatur anzufragen.
%
\item Ist das Fahrzeug repariert, soll das System fähig sein, das reparierte Fahrzeug wieder der Disposition zur Verfügung  zu stellen.
%
\item Das System soll Bestellungen auch von der Disposition empfangen.
%
\item Das System muss der Disposition die Möglichkeit bieten die freien Fahrzeuge anzuzeigen.
%
\item Hat ein Kunde ein Fahrzeug reserviert, soll das System ihm die Möglichkeit bieten, seine Reservation anzusehen.
\end{itemize}
\newpage
\section{Diskussion der Satzschablone}
Nachfolgend werden Vor- und Nachteile der Satzschablone erwähnt. Sie enstanden aus einer kurzen Diskussion.\\[2ex]
\textbf{Vorteile}
\begin{itemize}
\item Die Schablone hilft Ziele einfach und kurz zu formulieren.
\item Alle Ziele sind dann mehr oder weniger einheitlich formuliert. Es gibt dem ganzen eine Struktur.
\end{itemize}
%
\textbf{Nachteile}
\begin{itemize}
\item Komplexe Abläufe lassen sich nur schwer formulieren. (Aufbrechen in Teilaufgaben)
\end{itemize}

\end{document}
