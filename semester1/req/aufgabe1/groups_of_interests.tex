\section{Erhebung der Informationen}
Für die Erhebung der einzelnen Informationen gehen wir von den gleichen Gruppen aus, wie bei den Interessengruppen.
\subsection{Anwender}
Für die Anwender empfehlen wir einen Workshop. Dieser soll aufzeigen, in welche Richtung sich das Projekt entwickeln sollte, beziehungsweise welche Aufgaben die zukünftige Applikation erledigt.
%
\subsection{IT}
Das bestehende System gilt es zu analysieren. Um Probleme zu eliminieren, empfehlen wir durch System-Archäologie auch weitere Anforderungen frühzeitig zu erkennen.
% Nico ToDo: Ist die IT wirklich so wichtig???
Nach dem einlesen in verschiedene Dokumentationen, sollten wir ein Prototyp mit der IT erarbeiten. Die IT kennt das System und weiss wo wir es verbessern könnten. Wir gehen davon aus, dass die IT schon weiss, in welche Richtung sie die Webapplikation haben möchten. Mit einem Prototyp können wir schon ein bisschen den Aufwand abschätzen.
%
\subsection{Rechnungswesen}
Hier schlagen wir einen Workshop vor, der die Anwendungsfälle einbezieht. 
%Wir finden es wichtig, dass wir hier genau sehen, wie sich das ganze abspielt und wo wir mit der Applikation helfen können, die Kosten zu senken.
Dieser dient zum Erkennen der Abläufe um weitere Optimierungen in der Applikation vorzunehmen und damit Kosten zu senken.
%
\subsection{Disposition}
Die Disposition ist sehr negativ auf unser Projekt eingestimmt. Das müssen wir ändern. Wir finden, mit neuen kreativen Workshops können wir das ändern. Wichtig ist, dass wir ihnen zeigen können, dass sie 
auf das Projekt Einfluss nehmen können und sich einbringen müssen.
%mitbestimmen können bzw. müssen. 
%
\subsection{Direktion}
Der Direktor hat nicht viel Zeit, deshalb würden wir hier noch ein weiteres Interview mit einer Präsentation planen, um die nächsten Schritte zu besprechen und unsere Informationen, die wir von den Mitarbeiter gesammelt haben, zu präsentieren.\\
Zu diesem Zeitpunkt soll die Disposition von dem Projekt überzeugt sein. Was wir aber im Hinterkopf behalten müssen, ist, dass der Direktor in ein paar Jahren pensioniert wird.
%
\subsection{Kunden}
Bei den Kunden würden wir vorschlagen einen Fragebogen zu verwenden. Da es über 3000 Kunden sind, ist das die geeignetste Methode.\\
Es ist natürlich klar, dass man hier nie auf alle Kundenwünsche eingehen kann. Aber man kann mit einem guten Fragebogen die Richtung der Kunden herauslesen.