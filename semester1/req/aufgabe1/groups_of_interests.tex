% TODO: This is a draft! I have to rewrite it later ;-)
\section{Erhebung der Informationen}
\subsection{Anwender}
Für die Anwender schlagen wir eine Workshop vor. Dieser soll aufzeigen, in welche Richtung bzw. die zu Wünschende Aufgabe des Programms zum Vorschein bringen.
%
\subsection{IT}
Bei der IT können wir uns ein Prototyp vorstellen. Die IT kennt das System und weiss wo es hackt und wo man es verbessern könnte. Wir gehen davon aus, dass die IT schon weiss in welche Richtung sie die Webapplikation haben möchte. Hier geht es darum, um zu sehen ob es überhaupt machbar ist. Ein Prototyp würde schön zeigen, wie es sich die IT Vorstellt und die es man kann schon mal ein bisschen einschränken.
%
\subsection{Rechnungswesen}
Hier schlagen wir einen Workshop vor, der die Anwendungsfälle involviert. Wir finden es wichtig, dass wir hier genau sehen, wie sich das ganze abspielt und wo wir mit der Applikation helfen können, ihre Kosten zu senken.
%
\subsection{Disposition}
Die Disposition ist sehr negativ auf unser Projekt eingestimmt. Das müssen wir ändern. Wir finden, mit neuen kreativen Techniken können wir das ändern. Wichtig ist, dass wir ihnen zeigen können, dass sie mitbestimmen können bzw. müssen. 
%
\subsection{Direktion}
Der Direktor hat nicht viel Zeit, deshlab würden wir hier noch ein weiters Interview mit Presentation plannen, um die nächsten Schritte zu besprechen und unsere Informationen, die wir von den Mitarbeiter gesammelt haben, presentieren.\\
Zu diesem Zeitpunkt müssen wir es geschafft haben, die Disposition von dem Projekt zu überzeugen. Was wir aber im Hinterkopf behalten müssen, ist dass der Direktor in ein paar Jahren pensioniert wird. Ihm ist die gemütlichkeit bequemer als dass das Projekt ein Erfolg wird.
%
\subsection{Kunden}
Bei den Kunden würden wir vorschlagen einen Fragebogen zu verwenden. Da es über 3000 Kunden sind, ist das die geeignetste Methode.\\
Es natürlich klar das man hier nie auf alle Kundenwünsche eingehen kann. Aber man kann mit einem guten Fragebogen die Richtung der Kunden herauslesen.