\section{Erhebung der Informationen}
Für die Erhebung der einzelnen Informationen gehen wir von den gleichen Gruppen aus, wie bei den Interessengruppen.
\subsection{Anwender}
Für die Anwender schlagen wir eine Workshop vor. Dieser soll aufzeigen, in welche Richtung das ganze Projekt gehn sollte, beziehungsweise die zu wünschenden Aufgaben der zukünftigen Applikation.
%
\subsection{IT}
Hier müssen wir uns die System Archeologie genau anschauen. Somit können wir Probleme von vornherein eliminieren.\\
Nach dem einlesen, schlagen wir ein Prototyping mit der IT vor. Die IT kennt das System und weiss wo wir es verbessern könnten. Wir gehen davon aus, dass die IT schon weiss, in welche Richtung sie die Webapplikation haben möchten. Mit einem Prototyp können wir schon ein bisschen den Aufwand abschätzen.
%
\subsection{Rechnungswesen}
Hier schlagen wir einen Workshop vor, der die Anwendungsfälle involviert. Wir finden es wichtig, dass wir hier genau sehen, wie sich das ganze abspielt und wo wir mit der Applikation helfen können, ihre Kosten zu senken.
%
\subsection{Disposition}
Die Disposition ist sehr negativ auf unser Projekt eingestimmt. Das müssen wir ändern. Wir finden, mit neuen kreativen Workshops können wir das ändern. Wichtig ist, dass wir ihnen zeigen können, dass sie mitbestimmen können bzw. müssen. 
%
\subsection{Direktion}
Der Direktor hat nicht viel Zeit, deshlab würden wir hier noch ein weiters Interview mit einer Presentation plannen, um die nächsten Schritte zu besprechen und unsere Informationen, die wir von den Mitarbeiter gesammelt haben, zu presentieren.\\
Zu diesem Zeitpunkt müssen wir es geschafft haben, die Disposition von dem Projekt zu überzeugen. Was wir aber im Hinterkopf behalten müssen, ist, dass der Direktor in ein paar Jahren pensioniert wird.
%
\subsection{Kunden}
Bei den Kunden würden wir vorschlagen einen Fragebogen zu verwenden. Da es über 3000 Kunden sind, ist das die geeignetste Methode.\\
Es ist natürlich klar, dass man hier nie auf alle Kundenwünsche eingehen kann. Aber man kann mit einem guten Fragebogen die Richtung der Kunden herauslesen.