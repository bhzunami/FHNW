\section{Probleme und fehlende Informationen}

  \subsection{Bestehende Software}
  Der IT-Chef erwähnt, dass schon verschiedene kleinere Software-Lösungen im Einsatz stehen.
  Die Daten dieser Systeme müssen mit hoher Wahrscheinlichkeit in das neue System importiert werden können.
  Formate, Datenbank-Systeme, etc müssen abgeklärt und potenzielle Risiken dabei notiert werden.
  Falls möglich könnte man die UIs dieser Systemen nachbauen, um die Lernkurve der Endbenutzer zu reduzieren.

  \subsection{Excel}
  Laut Werkstattchef gibt es schon einen Excel-basierten Prozess, wodurch die Werkstatt mit dem Rechnungswesen interagiert.
  Es wäre von Vorteil die Vorlagen dieser Excel-Arbeitsblätter genauer analysieren zu dürfen um ein klareres Bild über den Informationsfluss zwischen den beiden Abteilungen zu schaffen.
  Dazu muss sämtliche Logik die in den Excel-Files eingebettet ist angeschaut werden um implizite Anforderungen zu erkennen und potenzielle Risiken zu identifizieren.
  
  \subsection{Interne Kommunikation}
  Der IT-Chef ist klar besser über Firmenweite Themen wie Freigaben von Budgets informiert als z.B. der Dispositionschef. Dies deutet auf ein Kommunkationsproblem hin.
  Man müsste also notieren, dass man nicht davon ausgehen kann, dass der Kader der Firma gross Informationen austauscht. Somit müssen alle Steakholders parallel informiert werden, eventuell durch verschiedene Kanäle.

  \subsection{Anwender-Segmentierung}
  Da die zu bauende Plattform ein Spektrumübergeifendes Problem löst werden alle involvierten Abteilungen andere Erwartungen des Endprodukts haben.
  Ideal wäre ein Gespühr für diese Benutzer-Segmente aufzubauen und die Endsoftware in einem modularen Stil aufbauen. Sprich, jeder Abteilung eine auf sie angepasste UI zu bieten.
  Gewisse Abteilungen werden auf einem höher-abstrahierten Niveau arbeiten als Andere.
  
  \subsection{Internationalisierung}
  Nicht alle Anwender der Applikation befinden sich umbedingt in der Deutschschweiz. Es müsste abgeklärt werden auf welche Sprachen die neue Lösung zu Verfügung stehen muss.
