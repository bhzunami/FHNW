\documentclass[12pt,article]{memoir}
\usepackage[utf8]{inputenc}
\usepackage[T1]{fontenc}
\usepackage{mathpazo}
\linespread{1.05}
\usepackage{tikz}
\usepackage{tikz-uml} 
\usepackage{amsmath}
\usepackage{savetrees}
\usepackage{multirow}

\begin{document}
\begin{center}
\begin{figure}[h!]
\begin{tikzpicture}
\begin{umlsystem}{Autoverwaltung Nutz AG}
% usecase-1
\umlusecase[x=0,y=6]{\#1: Fahrzeuge auflisten}
%
% usecase-2
\umlusecase[x=0,y=0]{\#2: Faktura erstellen}
%
% usecase-3
\umlusecase[x=0,y=2]{\#3: Statistik von Fahrzeugen anzeigen}
% 
% usecase-4
\umlusecase[x=0,y=4]{\#4: Fahrzeuge suchen}
%
% usecase-5
\umlusecase[x=0,y=8]{\#5: Fahrzeug reservieren}
%
% usecase-6
\umlusecase[x=0,y=-2]{\#6: Fahrzeug orten}
%
% usecase-7
\umlusecase[x=0,y=-4]{\#7: Fahrzeuge verwalten}
%
\end{umlsystem}

\umlactor[x=-9,y=9]{Kunde}
\umlactor[x=-7,y=0]{Disposition}
\umlactor[x=8,y=3]{Management}
\umlactor[x=-9,y=3]{Buchhaltung}

% Kunde
\umlassoc{Kunde}{usecase-5}
\umlassoc{Kunde}{usecase-1}
\umlassoc{Kunde}{usecase-4}

% Disposition
\umlassoc{Disposition}{usecase-3}
\umlassoc{Disposition}{usecase-6}
\umlassoc{Disposition}{usecase-7}

\umlassoc{Buchhaltung}{Kunde}
\umlassoc{Disposition}{Kunde}

% Buchhaltung
\umlassoc{Buchhaltung}{usecase-2}
\umlassoc{Buchhaltung}{usecase-3}

% Management
\umlassoc{Management}{usecase-3}

\end{tikzpicture}
\caption{Use Case Diagramm für Nutz AG}
\end{figure}
\end{center}



\begin{tabular}{|l|p{13cm}|}
	\hline
	ID & 5\\
	\hline
  Name & Fahrzeug reservieren\\
  \hline
  Beschreibung & Ein Fahrzeug wird von einem Kunden direkt oder über die Disposition reserviert. \\
  \hline
  Actors & Kunde oder Disposition\\
  \hline
  Vorbedingungen & Fahrzeug muss während dieser Zeit verfügbar sein. Das Fahrzeug muss im fahrtauglichen Zustand sein.\\
  \hline
  \multirow{3}{*}{Normalablauf} & 1. Kunde wählt Fahrzeug aus \\
  & 2. Kunde reserviert Fahrzeug \\
  & 3. Kunde erhält Reservationsbestätigung \\
  \hline
  Nachbedingungen & Fahrzeug hat das Flag reserviert und die Disposition wird über die Buchung informiert.\\
  \hline
  Alternativabläufe & Falls Fahrzeug reserviert oder voraussichtlich in Reparatur ist, wird die Reservation abgebrochen. \\
  \hline
\end{tabular}
\end{document}
