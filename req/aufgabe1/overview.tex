\section{Managment Summary}
Für die Firma Nutz AG soll eine neue Applikation entwickelt werden, um die Vermietung und Wartung der Fahrzeuge zielgerichteter und effizienter anzubieten.\\
Zurzeit besitzt die Firma ungefähr 200 verschiedene Nutzfahrzeuge. Ihre Kundendatenbank umfasst über 3000 Privat- sowie auch Geschäftskunden.\\
Durch die verschiedenen Standorte, kommt es immer wieder zu unnötigen Fahrzeugverschiebungen. Auch die Wartung der Fahrzeuge wird momentan ineffizient mit einer Excel Tabelle rapportiert.\\[2ex]
%
Dies soll durch die neue Webapplikation verbessert und vereinfacht werden.\\
%
Im ersten Kapitel werden die Stakeholder nach ihrer Wichtigkeit aufgelistet und in Interessengruppen eingeteilt.\\
Danach wird auf Probleme sowie Widerstände eingegangen, die im Verlauf des Projektes auftreten könnten.\\
Um an die richtigen Informationen zu gelangen, zeigen wir für jede Interessengruppe unsere Idee für die \textit{Erhebung der Informationen} auf.\\
Durch das Kontextdiagramm veranschaulichen wir den ganzen Arbeitsprozess\\
Im Kapitel 5 definieren wir die Projektziele, die wir erreichen möchten.\\
Im letzte Kapitel wird auf die Vor- und Nachteile der vorher angewendeten Satzschablonen eingegangen.