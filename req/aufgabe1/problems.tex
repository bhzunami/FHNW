\subsection{Probleme und fehlende Informationen}
%
  \textbf{Bestehende Software}\\
  Der IT-Chef erwähnt, dass zurzeit verschiedene kleinere Software-Lösungen im Einsatz stehen.
  Die Daten dieser Systeme müssen mit hoher Wahrscheinlichkeit in das neue System importiert werden können.
  Formate, Datenbank-Systeme und weiteres müssen zu gegebener Zeit abgeklärt und potenzielle Risiken dabei notiert werden.
  Falls nötig könnten die UIs dieser Systemen nachgebaut werden, um die Lernkurve der Endbenutzer zu reduzieren.\\[2ex]
%
  \textbf{Excel}\\
  Laut Werkstattchef gibt es schon einen auf Excel basierenden Prozess, mit dem die Werkstatt mit dem Rechnungswesen interagiert.
  Es wäre von Vorteil die Vorlagen dieser Excel-Arbeitsblätter genauer analysieren zu dürfen um ein klareres Bild über den Informationsfluss zwischen den beiden Abteilungen zu schaffen.
  Dazu muss sämtliche Logik die in den Excel-Files eingebettet ist analysiert werden um implizite Anforderungen zu erkennen und potenzielle Risiken zu identifizieren.\\[2ex]
  %
  \textbf{Interne Kommunikation}\\
  Der IT-Chef ist klar besser über Firmenweite Themen wie Freigaben von Budgets informiert als z.B. der Dispositionschef. Dies deutet auf ein Kommunikationsproblem hin.
  Man müsste also notieren, dass man nicht davon ausgehen kann, dass das Kader der Firma gross Informationen austauscht. Somit müssen alle Stakeholders parallel informiert werden, eventuell durch verschiedene Kanäle.\\[2ex]
%
  \textbf{Anwender-Segmentierung}\\
  Da die zu bauende Plattform ein spektrumübergeifendes Problem löst werden alle involvierten Abteilungen andere Erwartungen des Endprodukts haben.
  Ideal wäre ein Gespür für diese Benutzer-Segmente aufzubauen und die Endsoftware in einem modularen Stil aufzubauen. Sprich, jeder Abteilung eine auf sie angepasstes UI zu bieten.
  Gewisse Abteilungen werden auf einem höher-abstrahierten Niveau arbeiten als andere.\\[2ex]
  %
  \textbf{Internationalisierung}\\
  Möglicherweise befinden sich nicht alle Anwender der Applikation in der Deutschschweiz. Es müsste abgeklärt werden in welchen Sprachen die neue Lösung zu Verfügung stehen muss.
